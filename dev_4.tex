%dev_4.tex
%Par Guillaume Lahaie
%LAHG04077707
%
%%%%%%%%%%%%%%%%%%%%%%%%%%%%%%%%%%%%%%%%%
% Simple Sectioned Essay Template
% LaTeX Template
%
% This template has been downloaded from:
% http://www.latextemplates.com
%
% Note:
% The \lipsum[#] commands throughout this template generate dummy text
% to fill the template out. These commands should all be removed when 
% writing essay content.
%
%%%%%%%%%%%%%%%%%%%%%%%%%%%%%%%%%%%%%%%%%

%----------------------------------------------------------------------------------------
%	PACKAGES AND OTHER DOCUMENT CONFIGURATIONS
%----------------------------------------------------------------------------------------

\documentclass[10.9pt]{article} % Default font size is 12pt, it can be changed here
\renewcommand{\familydefault}{\rmdefault}
\renewcommand{\thesubsection}{\alph{subsection}}

%Pour l'encodage avec accents
\usepackage[utf8]{inputenc}
\usepackage{longtable}
\usepackage{algorithm2e}

%\usepackage{helvet}
%\renewcommand{\familydefault}{\sfdefault}

%Pour INF4100 - devoir 4
\usepackage{amsfonts}
\usepackage{mathtools}
\usepackage{multirow}

\usepackage{afterpage}
\usepackage{appendix}
\usepackage{graphicx} % Required for including pictures
\usepackage{listings}
\usepackage{mathtools}

\usepackage[left=2.2cm,top=2.2cm,right=2.2cm,bottom=2.2cm,nohead]{geometry} % Required to change the page size to A4
\geometry{letterpaper} % Set the page size to be A4 as opposed to the default US Letter

\usepackage{float} % Allows putting an [H] in \begin{figure} to specify the exact location of the figure
\linespread{1.2} % Line spacing

%\setlength\parindent{0pt} % Uncomment to remove all indentation from paragraphs

\graphicspath{{./Pictures/}} % Specifies the directory where pictures are stored
\usepackage[french,english]{babel}

%Comportement d'un paragraphe
\setlength{\parskip}{\baselineskip}%
\setlength{\parindent}{0pt}%

%Widows/orphans
\widowpenalty10000
\clubpenalty10000

\usepackage[hidelinks]{hyperref}

%Meta-info
\title{INF4100 - devoir 4}
\author{Guillaume Lahaie}
\date{Remise: 18 avril 2014}

\hypersetup{
  pdftitle={INF4100 - devoir 4},
  pdfauthor={Guillaume Lahaie}
}

\newcommand\blankpage{%
  \null
  \thispagestyle{empty}%
  \addtocounter{page}{-1}%
  \newpage}

\begin{document}
\selectlanguage{french}
\fussy

%----------------------------------------------------------------------------------------
%	TITLE PAGE
%----------------------------------------------------------------------------------------

\begin{titlepage}

\newcommand{\HRule}{\rule{\linewidth}{0.5mm}} % Defines a new command for the horizontal lines, change thickness here

\center % Center everything on the page

\textsc{\LARGE Université du Québec à Montréal}\\[1.5cm] % Name of your university/college
\textsc{\Large INF4100}\\[0.5cm] % Major heading such as course name

\HRule \\[1.5cm]
{ \huge \bfseries Devoir 4}\\[0.4cm] % Title of your document
\HRule \\[1.5cm]

\begin{minipage}{0.4\textwidth}
\begin{flushleft} \large
\emph{Par:}\\
Guillaume Lahaie \\ LAHG04077707 % Your name
\end{flushleft}
\end{minipage}
~
\begin{minipage}{0.4\textwidth}
\begin{flushright} \large
\emph{Remis à:} \\
Louise Laforest % Supervisor's Name
\end{flushright}
\end{minipage}\\[4cm]

{\large \emph{Date de remise:} \\ Le 18 avril 2014}\\[3cm] % Date, change the \today to a set date if you want to be precise

%\includegraphics{Logo}\\[1cm] % Include a department/university logo - this will require the graphicx package

\vfill % Fill the rest of the page with whitespace

\end{titlepage}
\blankpage

%----------------------------------------------------------------------------------------
%	TABLE OF CONTENTS
%----------------------------------------------------------------------------------------

\tableofcontents % Include a table of contents

\newpage % Begins the essay on a new page instead of on the same page as the table of contents 

%----------------------------------------------------------------------------------------
% SECTIONS DU DOCUMENT
%----------------------------------------------------------------------------------------


\section{Numéro 1.}

\subsection{Valeur à C[1, j], $0\le j \le L$}


Il y a deux valeurs possibles à C[1, j]. Soit il est possible de faire la monnaie exacte 
avec $P_1$, soit il n'est pas possible de faire la monnaie, est donc C[1, j] est l'infini.

\begin{equation}
C[1,j] = 
\begin{cases}
 \frac{j}{P_1}  &\text{si } j \bmod P_1 = 0\\
 \infty                   &\text{sinon}\\
\end{cases}
\end{equation}

\subsection{Valeur de C[i, 0], $1 \le i \le n$}

Comme la monnaie à rendre est de 0, à chaque valeur de C[i, 0] sera 0 car il n'y a pas
de monnaie à rendre.

\begin{equation}
 C[i, 0] = 0, 1 \le i \le n
\end{equation}

\subsection{Valeur de C[i, 1], $1 \le i \le n$}

Ici, la monnaie à rendre est de 1, donc on a deux valeurs possibles: soit l'ensemble de pièces
$P_1, P_2, ... , P_i$ contient une pièce de 1 sous, la valeur est alors 1, sinon la valeur est
l'infinie, il est impossible de rendre la monnaie.

\begin{equation}
C[i,1] = 
\begin{cases}
 1  &\text{si }  1 \in P_1, P_2, ... , P_i \\
 \infty                   &\text{sinon}\\
\end{cases}
\end{equation}

\subsection{Tableau C pour $P = [1, 5, 10, 25]$ et $L = 25$}
\begin{tabular}{c|c|c|c|c|c|c|c|c|c|c|c|c|c|c|c|c|c|c|c|c|c|}
\cline{2-22}
 $j=$&  0 & 1 & 2 & 3 & 4 & 5 & 6 & 7 & 8 & 9 & 10 & 11 & 12 & 13 &14 & 15 & 16 & 17 & 18 & 19 & 20 \\ \cline{2-22}
\multirow{4}{*}{$C=$} & 0 & 1 & 2 & 3 & 4 & 5 & 6 & 7 & 8 & 9 & 10 & 11 & 12 & 13 &14 & 15 & 16 & 17 & 18 & 19 & 20 \\ \cline{2-22}
 & 0 & 1 & 2 & 3 & 4 & 1 & 2 & 3 & 4 & 5 & 2 & 3 & 4 & 5 & 6 & 3 & 4 & 5 & 6 & 7 & 4 \\ \cline{2-22}
 & 0 & 1 & 2 & 3 & 4 & 1 & 2 & 3 & 4 & 5 & 1 & 2 & 3 & 4 & 5 & 2 & 3 & 4 & 5 & 6 & 2 \\ \cline{2-22}
 & 0 & 1 & 2 & 3 & 4 & 1 & 2 & 3 & 4 & 5 & 1 & 2 & 3 & 4 & 5 & 2 & 3 & 4 & 5 & 6 & 2 \\ \cline{2-22}
\end{tabular}

\subsection{Équation de récurrence pour C[i, j]}

\begin{equation}
 C[i, j] = 
 \begin{cases}
  0 			&\text{si } j = 0 \\
  \frac{j}{P_1} 	&\text{si } i = 1, 1 \le j \le L, j \bmod P_1 = 0 \\
  \infty 		&\text{si } i = 1, 1 \le j \le L, j \bmod P_1 \neq 0 \\
  C[i-1, j]		&\text{si } j < P_i \\
  min(1 + C[i, j - P_i]), C[i-1, j])	&\text{sinon}
 \end{cases}
\end{equation}

\subsection{Algorithme de programmation dynamique utilisant un tableau en une dimension}

\begin{algorithm}
 \SetKwInput{Fonction}{fonction}
 \SetKwInput{Donnees}{donnees}
 \SetKwInput{Antecedents}{antécédents}
 \SetKwInput{Consequents}{conséquents}
  \Fonction{monnaieOptimale ($P, L$)}
 \Donnees{ \emph{P}: tableau des pièces de monnaie indicé de 1 à $n$ \\ 
	   \emph{L}: la monnaie à rendre}
 \Sortie{ \emph{C}: le tableau de la monnaie optimale de 0 à $L$}
   \Antecedents{$P[1] \le P[2] \le ... \le P[n]$}
 \Deb{
    créer tableau $C$ de longueur $L+1$ et initialiser les valeurs à $\infty$\\
    \Pour{$j \longleftarrow 1$ à $n$}{
      $C[0] = 0$\\
      \Pour{$i \longleftarrow 1$ à $L$}{
	\Si{$P[j] \le i$}{
	  $C[i] \longleftarrow min(C[i], 1 + C[i-P[j]])$
	}
      }
    }
  }     
\end{algorithm}

\subsection{Complexité temporelle}

La complexité dépend du nombre de pièces disponibles pour faire la monnaie, et $L$, la monnaie
à rendre.

On a deux boucles. On boucle tout d'abord sur les pièces, il y a donc $n$ itérations.
À chaque itération, on vérifie pour chaque valeur de $0$ à $L$ si on a une meilleure solution.
Donc, on a pour cette boucle intérieure on a $L$ itérations. 

On fait un nombre constant d'opérations à l'intérieur des deux boucles. Donc, 
l'algorithme à une complexité temporelle de $\Theta(nL)$.
\newpage
\subsection{Algorithme glouton capable de faire la monnaie}

\begin{algorithm}
 \SetKwInput{Fonction}{fonction}
 \SetKwInput{Donnees}{donnees}
 \SetKwInput{Antecedents}{antécédents}
 \SetKwInput{Consequents}{conséquents}
  \Fonction{trouverChangeOptimale ($P, L, C$)}
 \Donnees{ \emph{P}: tableau des pièces de monnaie indicé de 1 à $n$ \\ 
	   \emph{L}: la monnaie à rendre\\
	   \emph{C}: le tableau de la monnaie optimale de 0 à $L$}
  \Antecedents{$M \le L$\\
	       $P[1] \le P[2] \le ... \le P[n]$\\
	       Il existe une solution pour monnaie $(C[monnaie] \neq \infty)$}
 \Sortie{$S$, tableau contenant le nombre de pièces à rendre pour chaque pièce de monnaie, $S[0]$ représente
	  le nombre de pièces de $P[0]$ nécessaires, etc.}
 \Deb{
  Créer un tableau $S$ de longueur $n$, et initialiser toutes les valeurs à $0$\\
  $i \longleftarrow n$\\
  \Tq{$L > 0$}{
    \Si{$L \ge P[i]$ et $C[L] = 1 + C[L-P[i]]$}{
      $L \longleftarrow L - P[i]$\\
      $S[i] \longleftarrow S[i] + 1$\\
      }
    \Sinon{
      $i \longleftarrow i - 1$
      }
    }
    \Retour{$S$}
    }
\end{algorithm}

\subsection{Complexité temporelle de l'algorithme précédent}

On peut d'abord étudier le pire cas cet algorithme. Cela se produit lorsqu'on fait le 
nombre maximale de la boucle Tant que.

Dans ce cas, on fait $n + \frac{L}{P_1}$ tours de boucle: la monnaie maximale contient $L$ pièces
de valeur $P_1$. On a donc vérifier toutes les pièces de plus grande valeur que $P_1$, ensuite
on va itérer jusqu'à ce que $L$ soit égale à 0, donc dans ce cas $\frac{L}{P_1}$. 

La complexité temporelle du pire cas est donc $\Theta(n + L)$.

Dans le meilleur cas, on a seulement besoin des pièces $P_n$ pour faire la monnaie. Il n'est donc
pas nécessaire de tester les autres pièces.


\end{document}